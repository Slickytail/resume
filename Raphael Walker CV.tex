% Raphael Walker Resume
\documentclass[12pt,letterpaper]{article}
\usepackage{geometry}
\geometry{
    letterpaper,
    top=0.6in,
    bottom=0.4in,
    left=0.5in,
    right=0.5in
}
\usepackage{mdwlist}
% Fonts
\usepackage[utf8]{inputenc}
\usepackage[english]{babel}
\usepackage{lmodern}
\usepackage[T1]{fontenc}
\usepackage[babel=true]{microtype}
\usepackage{xcolor}

\usepackage{textcomp}
\usepackage{enumitem}
\usepackage{fontawesome}
\usepackage{multicol}
% Hyperlink setup
\usepackage{hyperref}
\usepackage{graphicx}
% Math stuff
\usepackage{amsmath,amsfonts}

\pagestyle{empty}
\setlength{\tabcolsep}{0em}

\newenvironment{explist}
{\begin{itemize}[label=\textbf{--},itemsep=1pt,topsep=0pt,partopsep=0pt,parsep=0pt]}
{\end{itemize}}

\begin{document}

%    \begin{multicols}{2}

{\LARGE{\textbf{Raphaël Barish Walker}}}\vspace{7pt}

\faPhone  +33 06 95 87 96 73 (FR), +1 (510) 847-3889 (US)\vspace{4pt}

\faEnvelope raf@raphaelwalker.com\vspace{4pt}

\faGithub \href{https://github.com/Slickytail}{github.com/Slickytail}\vspace{4pt}

\faLinkedinSquare \href{https://linkedin.com/in/raphael-walker}{linkedin.com/in/Raphael-Walker}\vspace{4pt}

%        \columnbreak
%        \begin{flushright}
%            \includegraphics[width=0.3\linewidth]{headshot.jpg}
%        \end{flushright}
%    \end{multicols}

\hrule

\section*{Publications}
\begin{itemize}[label=]
    \item
        \textbf{Benchmarking the Borg algorithm on the bbob-biobj testbed}\\
        \textit{GECCO '23: BBOB Workshop \hfill 2023}\\
        \textit{With D. Brockhoff, P. Capetillo, and J. Hornewall}

        The Borg MOEA is an optimization algorithm, designed to handle real-world problems of a multi objective and multimodal nature.
        We benchmark the performance of the algorithm on the bbob-biobj test suite via the COCO platform, comparing it to current state-of-the-art algorithms.
        We also develop a custom parameter tuning scheme, which substantially improves Borg's performance on the test suite without problem-specific information.

    \item
        \textbf{How Many Cards Should You Lay Out in a Game of EvenQuads: A Detailed Study of Caps in AG$(n, 2)$}\\
        \textit{La Matematica, Vol. 2 No. 2 \hfill 2023}\\
        \textit{With J. Crager, F. Flores, T. Goldberg, L. Rose, D. Rose-Levine, and D. Thornburgh}

        We define a cap in the affine geometry AG$(n, 2)$ to be a subset in which any collection of 4 points is in general position.
        We classify, up to affine equivalence, all caps in AG$(n, 2)$ of size at most $k \leq 9$.

    \item
        \textbf{A Small Maximal Sidon Set in $\mathbb{Z}_2^n$}\\
        \textit{Siam Journal of Discrete Mathematics, Vol. 36 No. 3 \hfill 2022}\\
        \textit{With M. Redman and L. Rose}

        A Sidon set is a subset of an Abelian group with the property that each sum of two distinct elements is distinct.
        We construct a small maximal Sidon set of size $O((n \cdot 2^n)^{1/3})$ in the group ${\mathbb{Z}}_2^n$,
        generalizing a result of Ruzsa concerning maximal Sidon sets in the integers.

    \item
        \textbf{Lagrangian Cobordisms of Legendrian Pretzel Knots with Maximal Thurston-Bennequin Number}
        \textit{Undergraduate Senior Thesis, Bard College \hfill 2021}
        \textit{With C. Leverson}

        In the study of Legendrian knots, which are smoothly embedded circles constrained by a differential geometric condition,
        an actively-studied problem is to find conditions for the existence of Lagrangian cobordisms,
        which are Lagrangian surfaces whose slices resemble specific Legendrian knots at each end.
        We demonstrate a family of knots where each has a maximal-tb representative admitting a Lagrangian cobordism from a stabilized Legendrian unknot.

\end{itemize}

\section*{Academic Experience}
\begin{itemize}[label=]
    \item
        \textbf{Université Paris-Saclay} \textit{Orsay, France} \hfill 2023\\
        \textit{Master of Science, Mathematics of Artificial Intelligence}

    \item
        \textbf{INRIA} \textit{Saclay, France} \hfill Summer 2023\\
        \textit{CELESTE Team, Research Intern}
        \textit{Advisor: \'Etienne Boursier}
        \begin{explist}
        \item Studied the convergence of gradient descent for shallow neural networks.
        \item Investigated the phenomenon where classification problems are ``easier to solve'' than regression problems with the same data.
        \item Found simple examples where infinitely-wide neural networks converge to a suboptimal local minimum.
        \end{explist}

    \item
        \textbf{Bard College} \textit{Annandale-On-Hudson, New York} \hfill Spring 2021\\
        \textit{Bachelor of Arts, Mathematics Major}\hfill \textit{GPA 4.0}\\
        Artine Artinian Scholar 2019--2020\\
        Mathematics and computer science tutor

    \item
        \textbf{Bard Summer Research Institute} \hfill Summer 2021\\
        \textit{Research Assistant, Mathematics}\\
        \textit{Mentor: Prof. Lauren Rose}
        \begin{explist}
        \item Investigated the size of maximal and minimal generalized caps in finite affine spaces.
        \item Introduced other students in the research group to the material and supported their exploration of related problems.
        \item Wrote programs to compute cap sizes in specific affine spaces through optimized brute-force search.
        \item Created \href{https://slickytail.github.io/QuadsVis/index.html}{a webapp to visualize generalized affine caps}.
        \end{explist}

    \item
        \textbf{Bard College at Simon's Rock} \textit{Great Barrington, Massachusetts} \hfill Spring 2019\\
        \textit{Associate of Arts} \hfill \textit{GPA 3.9}\\
        Mathematics, computer science, and French tutor\\
        Dean's List

\end{itemize}

\section*{Work Experience} \vspace{-0.5em}
\begin{itemize}[label=]
    \item
        \textbf{Flim} \hfill October 2023---\\
        \textit{Machine Learning Research Scientist}
        \begin{explist}
        \item Assisted with a collaboration between visual artists G\'erard Garouste and Neil Beloufa, using machine learning models trained on their artworks.
        \item Built a generative AI system for ``combining'' pairs of images.
            Applied cutting-edge fine-tuning objectives to create and insert custom layers into foundation models.
        \item Built a pipeline to extract the most aesthetically pleasing images from a video.
            Trained a custom preference model on a large-scale internal dataset,
            and performed intensive optimization to enable efficient CPU inference.
        \end{explist}

    \item
        \textbf{Invisible College} \hfill 2021---2023\\
        \textit{HCI Researcher}
        \begin{explist}
        \item \href{https://peeryview.org}{PeeryView.org}
        \item Prototyped and built an online tool implementing decentralized and subjective peer review, and archival and discussion of web links.
        \item Collaborated with the PeeryView design team to determine the tooling needs of the scientific community.
        \item Server as ML/science advisor to a psychiatric team building a prototype of an LLM-based cognitive behavioral therapy program.
        \end{explist}
    \item
        \textbf{Invisible College} \hfill Summers 2019, 2020\\
        \textit{Research Assistant}
        \begin{explist}
        \item Designed and developed a set of decentralized synchronization protocols and algorithms.
        \item Co-authored IETF draft for universal synchronization protocol.
        \item Created Javascript and NodeJS tools to analyze and debug synchronization algorithms, including a universal protocol translation demo and a peer-to-peer sync visualization.
        \item Contributed to client and server code for the \href{https://github.com/braid-work/braidjs}{BraidJS} library.
        \end{explist}

    \item
        \textbf{Speakeasy Digital Media} \hfill September---December 2018\\
        \textit{Web Developer} \hfill Part-time
        \begin{explist}
        \item Created and modified WordPress PHP templates for company blogs.
        \item Improved page load times by up to ten times by optimization on both front-end page loading and back-end content generation.
        \end{explist}

    \item
        \textbf{Storefront Political} \hfill Summer 2018\\
        \textit{Data Science Intern}
        \begin{explist}
        \item Analyzed pre-electoral polls, including weighting, cross-tabulating, raking, and cleaning.
        \item Created R and Python scripts to automate common tasks such as matching ZIP codes to voting districts and visualizing survey results.
        \item Designed and implemented a webapp for interactive visualization of survey results.
        \item Managed large PostgreSQL databases containing voter information.
        \end{explist}

    \item
        \textbf{Omnisparx} \hfill November 2017---April 2018\\
        \textit{Intern}
        \begin{explist}
        \item Researched and reported on the state of blockchain technology to educate app users and inform development for blockchain startup.
        \item Reported directly to CEO.
        \end{explist}

\end{itemize}

\end{document}
